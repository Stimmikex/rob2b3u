\section{Inngangur}

Verkefni: \\
Verkefnið sem ég er að vinna á er að gera gasmæli sem getur verið létt að ferðast með og var hugsunin mín að gera hann eins lítinn eins og ég gæti.. Verkefnið er eiginlega skift í skref þar sem ég byrja á einum parti af verkefninu eins og gas mælinum, athuga hvort hann virkar og ef hann virkar fer ég að prófa annan part af verkefninu og ég geri þetta þangað til að allt verkefnið er tilbúið og allir hlutar af verkefninu virka vel stakir og saman, allir hlutir sem ég er að nota í verkefninu eru components sem virka með Arduino. \\

Forritunar mál: \\
Verkefnið er skrifað í C (vélar kóða) og er þessi kóði notaður í gegnum Arduino tölvu, talvan les inn kóðan sem er skiftur í marga hluta eða eftir hverjum component fyrir sig. Arduino er lítill talva sem er gerð til að get verið notuð í allskyns verkefni eins og þetta sem ég er að gera, talvan sjálf er kanski ekki mikill en allt sem er í kringum hana er og það er hægt að bæta endalausum hlutum/components við hana. \\

Notkun: \\
Þessi gas mælir er beint að þeim sem eru að ferðast mikið og eru að nota gas brennara í sumarbústaðinum eða bara húsbílnum jaft fram getur þessi gasmælir notaður í bílskúrum og bara hvar sem er þar sem það hentar að hafa vörn við gasleka þar sem notkun á hættulegu gas á sér staðar. Þessi mælir mun líka vera með fleiri not t.d. eins og Hita/Rakamæli sem er tengdur við skjáinn í verkefninu var sem notandi getur séð hvað mikill raki er í loftinu og líka hitastigið þar sem mælirinn er. \\

Hugsanleg bæting: \\
Ég var búinn að hugsa mér að geta gert tvenskonar mæli þar sem ein er gerður fyrir heima/bústaðar notkun og annar getur fyrir úti notkun, mælirninn sem er getur fyrir heima/bústaðar notkun gæti verið tengdur við internetið og með því getur mælirinn sent uplýsingar um gasleka og líka uplýsingar um raka og hita í húsinu. Hinn mælirinn væri getur til þess að vera létt að taka með sér og mundi þurfa aðra leið á því að senda skilaboð, þá mundi ég nota GSM til þess að senda skilaboð til notenda og þetta getur verið notað til þess að segja frá um gaslek og hita/rakastig alveg eins og hinn nema að heim mælirinn mundi vera meira tengdur og líklegast stungið í samband við rafmagn þar sem útimælirnn mundi vera með batteríi. \\

Hér er listi yfir vefi sem ég notað i\\
http://www.electroschematics.com \cite{DHT22}
http://www.circuitstoday.com/gas-leakage-detector-using-arduino-with-sms-alert \cite{Gasleak}
https://learn.sparkfun.com/tutorials/hazardous-gas-monitor \cite{Gasmon}

\begin{figure}[h]
\end{figure}